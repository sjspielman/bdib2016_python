\documentclass{article}[12pt]
\usepackage[margin=0.75in]{geometry}
\usepackage{color}
\usepackage{graphicx}
%\usepackage{hyperref}
\usepackage{fancyvrb}
\usepackage{palatino}
\usepackage{enumitem}
\usepackage{natbib}
\usepackage[T1]{fontenc} % quotes
%\hypersetup{
%	colorlinks=true, %set true if you want colored links
%	linktoc=all,     %set to all if you want both sections and subsections linked
%	linkcolor=blue,  %choose some color if you want links to stand out
%	urlcolor=blue
%}
\newcommand{\code}[1]{\texttt{#1}}  % let's not use bold for now
\sloppypar
\setlength\parindent{0pt}
\usepackage{textcomp}

\begin{document}

\title{Introduction to Python \\ Day Three Exercises}
\author{Stephanie Spielman \\ \footnotesize{Email: stephanie.spielman@gmail.com}}
\date{}
\maketitle{}


\section{File Input and Output}

\textbf{Be sure to always close the file once you are done with it! If you use \code{with} control flow, Python will close the file automatically.}
The following exercises make use of files distributed in today's course materials. Be sure that you are in the correct directory when interacting with the files!

\begin{enumerate}
	
	\item Open the file "file1.txt" in read-mode, and print its contents to screen. For this, you should use the \code{.read()} method, which saves the contents of the file to a single string. Perform this task twice: once using \code{open} and \code{close}, and once using \code{with} control-flow.
	
	\item Open the file "file1.txt" in read-mode, and save all lines in this file to a list using the \code{.readlines()} method. Write a new file called "upper\_file1.txt" which contains the same contents of "file1.txt" but in upper-case. Try to do this task using a single for-loop, and don't forget that in order to write newlines (the "enter" key) to a file, you must include \code{"\textbackslash n"} in the string you are writing to file!
	
	\item Open the file "upper\_file1.txt" in append-mode, and \emph{append} the sentence: "I just created this upper-case file!" to the bottom of the file. Close the file and open it (separately) to examine the contents. 
	
	\item Again, open the file "upper\_file1.txt", this time in read-mode. Loop over the file lines \emph{without} using \code{.read()} or \code{.readlines()}. Print out lines as you loop.
	
	\item Modify the previous for-loop to only print out lines in "upper\_file1.txt" which contain at least (i.e. $>=$) 5 letter "E"s. This will require an \code{if} statement as well as the method \code{.count()}.
	
	\item You should notice 20 files named file1.txt, file2.txt, ...file20.txt. Write a for-loop to open each of these files (Hint: use the \code{range()} function to loop over file names). For each file, write a for-loop over the file lines (this does not use \code{.read()} or \code{.readlines()}). Print each line that contains more than 25 characters (determine this with \code{len()}.)
	
	\item Using the zoo-keeper dictionaries from the \textbf{Working with dictionaries} exercises above, create and write to a new file called "animal\_food.txt" which contains the sentences you created earlier: \\
        The lion eats meat \\
        The gazelle eats grass \\ 
        The anteater eats termites \\
        The alligator eats visitors \\ 
            ... etc. \\

\end{enumerate}



\section{Defining Functions}

For this set of exercises, you will re-write some of yesterday\textquotesingle s exercises as functions. The code doing the actual computation will remain virtually the same, except it will be written in the context of a function and subsequently called. \textbf{ After you write each function, run it with 2-3 test cases to confirm that it works as expected!!}

\begin{enumerate}

	\item In Texas, you can be a member of the elite "top 1\%" if you make at least \$423,000 per year. Alternatively, in Hawaii, you can be a member once you start making at least \$279,000 per year! Finally, if you live in New York, you need to earn at least \$506,000 a year to make the cut. \\\\ Write a function to determine if a given salary is a "top-1\%" salary in Texas, Hawaii, and/or New York. Your function should take a single argument, the salary, and \emph{print} a sentence indicating in which state(s) this salary is and is not a top-1\% salary. Your function should not return a value.
	
	\item Write a function that returns a list of the powers (exponents 0-15, inclusive!!) for a provided number. This function should take 1 argument: the number to raise to powers 0-15.
	
	\item Modify the previous function to take two additional arguments indicating which exponents to calculate. For example, in the previous exercise, you would have provided the additional arguments 0 and 15 (or, 16, depending on how you wrote your loop). 
	
	\item Write a function to curve a list of grades, silly-professor style. This function should take \emph{four arguments}:
	\begin{itemize}
		\item A list of grades to curve
		\item The cutoff \emph{above which} grades are reduced
		\item The cutoff \emph{below which} grades are raised
		\item The scaling value
	\end{itemize}
	The function should return a list of curved grades. 
	
	\item Write a function to compute the molecular weight of a protein sequence. This function should take a single argument, a protein-sequence string, and it should return a single value, the molecular weight. Your function should account for the potential presence of ambiguous amino acids (again, compute these weights from average of all weights). Once your function is written, run it on the protein sequence "ABVPOXIRBTQQWS." \\ Use this dictionary in your function:
	\\ \code{ amino\_weights = \{"A":89.09, "R":174.20, "N":132.12, "D":133.10, "C":121.15, "Q":146.15, "E":147.13, "G":75.07, "H":155.16, "I":131.17, "L":131.17, "K":146.19, "M":149.21, "F":165.19, "P":115.13, "S":105.09, "T":119.12, "W":204.23, "Y":181.19, "V":117.15\}} 
	
	\item Write a function to open file and return the number of letter "f"\textquotesingle s in the file (both capital and lower-case!). 

	\item Modify the previous function such that it contains an additional argument indicating which letter/character to count.
	
\end{enumerate}


\end{document}
